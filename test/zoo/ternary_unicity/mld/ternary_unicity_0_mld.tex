\documentclass[a4paper]{article}
\usepackage[normalem]{ulem}
\usepackage[T1]{fontenc}
\usepackage[french]{babel}
\frenchsetup{StandardLayout=true}

\newcommand{\relat}[1]{\textsc{#1}}
\newcommand{\attr}[1]{#1}
\newcommand{\prim}[1]{\uline{#1}}
\newcommand{\foreign}[1]{\#\textsl{#1}}

\title{Conversion en relationnel\\du MCD \emph{ternary\_unicity}}
\author{\emph{Généré par Mocodo}}

\begin{document}
\maketitle

\begin{itemize}
  \item \relat{Disponibilité} (\foreign{\prim{semaine}}, \prim{voilier})
  \begin{itemize}
    \item Le champ \emph{semaine} fait partie de la clé primaire de la table. C'est une clé étrangère qui a migré à partir de l'entité \emph{Semaine} pour renforcer l'identifiant.
    \item Le champ \emph{voilier} fait partie de la clé primaire de la table. Il a migré à partir de l'entité \emph{Voilier} pour renforcer l'identifiant. Cependant, comme la table créée à partir de cette entité a été supprimée, il n'est pas considéré comme clé étrangère.
  \end{itemize}

  \item \relat{Réservation} (\prim{id résa}, \attr{num résa}$^{u\_1}$, \attr{arrhes}, \attr{date réservation}, \foreign{semaine}$^{u\_2}$, \foreign{voilier}$^{u\_2}$)
  \begin{itemize}
    \item Le champ \emph{id résa} constitue la clé primaire de la table. C'était déjà un identifiant de l'entité \emph{Réservation}.
    \item Le champ \emph{num résa} était déjà un simple attribut de l'entité \emph{Réservation}. Il obéit à la contrainte d'unicité 1.
    \item Les champs \emph{arrhes} et \emph{date réservation} étaient déjà de simples attributs de l'entité \emph{Réservation}.
    \item Les champs \emph{semaine} et \emph{voilier} sont des clés étrangères. Ils ont migré par l'association de dépendance fonctionnelle \emph{DF} à partir de l'entité \emph{Disponibilité} en perdant leur caractère identifiant. Ils obéissent en outre à la contrainte d'unicité 2.
  \end{itemize}

  \item \relat{Semaine} (\prim{semaine}, \attr{date début}$^{u\_1}$)
  \begin{itemize}
    \item Le champ \emph{semaine} constitue la clé primaire de la table. C'était déjà un identifiant de l'entité \emph{Semaine}.
    \item Le champ \emph{date début} était déjà un simple attribut de l'entité \emph{Semaine}. Il obéit à la contrainte d'unicité 1.
  \end{itemize}

\end{itemize}

\end{document}
