\documentclass[a4paper]{article}
\usepackage[normalem]{ulem}
\usepackage[T1]{fontenc}
\usepackage[french]{babel}
\frenchsetup{StandardLayout=true}

\newcommand{\relat}[1]{\textsc{#1}}
\newcommand{\attr}[1]{#1}
\newcommand{\prim}[1]{\uline{#1}}
\newcommand{\foreign}[1]{\#\textsl{#1}}

\title{Conversion en relationnel\\du MCD \emph{inheritance}}
\author{\emph{Généré par Mocodo}}

\begin{document}
\maketitle

\begin{itemize}
  \item \relat{TRISTIS} (\prim{magna}, \attr{vestibulum}, \attr{type!}, \attr{convallis?}, \attr{ipsum?}, \attr{pulvinar?}, \attr{audis?}, \foreign{magna via\_mollis?}, \foreign{magna via\_vitae?}, \attr{tempor?}, \attr{fugit?})
  \begin{itemize}
    \item Le champ \emph{magna} constitue la clé primaire de la table. C'était déjà un identifiant de l'entité \emph{TRISTIS}.
    \item Le champ \emph{vestibulum} était déjà un simple attribut de l'entité \emph{TRISTIS}.
    \item Un discriminateur à saisie obligatoire \emph{type} est ajouté pour indiquer la nature de la spécialisation. Jamais vide, du fait de la contrainte de totalité.
    \item Le champ à saisie facultative \emph{convallis} a migré à partir de l'entité-fille \emph{SODALES} (supprimée).
    \item Le champ à saisie facultative \emph{ipsum} a migré à partir de l'entité-fille \emph{SODALES} (supprimée).
    \item Le champ à saisie facultative \emph{pulvinar} a migré à partir de l'entité-fille \emph{NEC} (supprimée).
    \item Le champ à saisie facultative \emph{audis} a migré à partir de l'entité-fille \emph{NEC} (supprimée).
    \item Le champ à saisie facultative \emph{magna via\_mollis} est une clé étrangère. Il a migré à partir de l'entité-fille \emph{NEC} (supprimée) dans laquelle il avait déjà migré à partir de l'entité \emph{TRISTIS}.
    \item Le champ à saisie facultative \emph{magna via\_vitae} est une clé étrangère. Il a migré à partir de l'entité-fille \emph{NEC} (supprimée) dans laquelle il avait déjà migré à partir de l'entité \emph{TRISTIS}.
    \item Le champ à saisie facultative \emph{tempor} a migré à partir de l'entité-fille \emph{LACUS} (supprimée).
    \item Le champ à saisie facultative \emph{fugit} a migré à partir de l'entité-fille \emph{LACUS} (supprimée).
  \end{itemize}

  \item \relat{ULTRICES} (\foreign{\prim{magna sodales}}, \foreign{\prim{magna lacus}})
  \begin{itemize}
    \item Les champs \emph{magna sodales} et \emph{magna lacus} constituent la clé primaire de la table. Ce sont des clés étrangères qui ont migré directement à partir de l'entité \emph{TRISTIS}.
  \end{itemize}

\end{itemize}

\end{document}
