\documentclass[a4paper]{article}
\usepackage[normalem]{ulem}
\usepackage[T1]{fontenc}
\usepackage[french]{babel}
\frenchsetup{StandardLayout=true}

\newcommand{\relat}[1]{\textsc{#1}}
\newcommand{\attr}[1]{#1}
\newcommand{\prim}[1]{\uline{#1}}
\newcommand{\foreign}[1]{\#\textsl{#1}}

\title{Conversion en relationnel\\du MCD \emph{triple\_NNN}}
\author{\emph{Généré par Mocodo}}

\begin{document}
\maketitle

\begin{itemize}
  \item \relat{Appliquer} (\prim{employé}, \prim{projet}, \prim{compétence})
  \begin{itemize}
    \item Le champ \emph{employé} fait partie de la clé primaire de la table. Sa table d'origine (\emph{Employé}) ayant été supprimée, il n'est pas considéré comme clé étrangère.
    \item Le champ \emph{projet} fait partie de la clé primaire de la table. Sa table d'origine (\emph{Projet}) ayant été supprimée, il n'est pas considéré comme clé étrangère.
    \item Le champ \emph{compétence} fait partie de la clé primaire de la table. Sa table d'origine (\emph{Compétence}) ayant été supprimée, il n'est pas considéré comme clé étrangère.
  \end{itemize}

\end{itemize}

\end{document}
