\documentclass[a4paper]{article}
\usepackage[normalem]{ulem}
\usepackage[T1]{fontenc}
\usepackage[french]{babel}
\frenchsetup{StandardLayout=true}

\newcommand{\relat}[1]{\textsc{#1}}
\newcommand{\attr}[1]{#1}
\newcommand{\prim}[1]{\uline{#1}}
\newcommand{\foreign}[1]{\#\textsl{#1}}

\title{Conversion en relationnel\\du MCD \emph{protected}}
\author{\emph{Généré par Mocodo}}

\begin{document}
\maketitle

\begin{itemize}
  \item \relat{EROS} (\prim{congue})
  \begin{itemize}
    \item Le champ \emph{congue} constitue la clé primaire de la table. C'était déjà un identifiant de l'entité \emph{EROS}.
  \end{itemize}

  \item \relat{LACUS} (\prim{blandit}, \attr{elit})
  \begin{itemize}
    \item Le champ \emph{blandit} constitue la clé primaire de la table. C'était déjà un identifiant de l'entité \emph{LACUS}.
    \item Le champ \emph{elit} était déjà un simple attribut de l'entité \emph{LACUS}.
  \end{itemize}

  \item \relat{LIGULA} (\foreign{\prim{blandit}}, \foreign{congue}, \attr{metus})
  \begin{itemize}
    \item \paragraph{Avertissement.} Table résultant de la conversion forcée d'une association DF.
    \item Le champ \emph{blandit} constitue la clé primaire de la table. C'est une clé étrangère qui a migré directement à partir de l'entité \emph{LACUS}.
    \item Le champ \emph{congue} est une clé étrangère. Il a migré directement à partir de l'entité \emph{EROS} en perdant son caractère identifiant.
    \item Le champ \emph{metus} était déjà un simple attribut de l'association \emph{LIGULA}.
  \end{itemize}

\end{itemize}

\end{document}
